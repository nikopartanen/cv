% !TEX encoding = UTF-8 Unicode

%----------------------------------------------------------------------------------------
%	PACKAGES AND OTHER DOCUMENT CONFIGURATIONS
%----------------------------------------------------------------------------------------

\documentclass[11pt, a4paper]{article}

\usepackage{xltxtra,fontspec,xunicode,graphicx,graphics,geometry,setspace,multicol,multirow}


\geometry{a4paper, textwidth=5.5in, textheight=8.5in, marginparsep=7pt, marginparwidth=.6in}
\setlength\parindent{0in}

\usepackage[usenames,dvipsnames]{xcolor}

\usepackage{linguex}
\usepackage{sectsty} 

\usepackage[normalem]{ulem} 
\usepackage{xunicode} 

\defaultfontfeatures{Mapping=tex-text} 

\usepackage{marginnote} % For margin years
\newcommand{\years}[1]{\marginnote{\scriptsize #1}} % New command for including margin years
\renewcommand*{\raggedleftmarginnote}{}
\setlength{\marginparsep}{7pt} % Slightly increase the distance of the margin years from the contant
\reversemarginpar

\usepackage[xetex, bookmarks, colorlinks, breaklinks, pdftitle={Niko Partanen - curriculum vitae},pdfauthor={Niko Partanen}]{hyperref} 

\hypersetup{linkcolor=blue,citecolor=blue,filecolor=black,urlcolor=MidnightBlue} 

%----------------------------------------------------------------------------------------
%	FONT CONFIGURATIONS
%----------------------------------------------------------------------------------------

% Two font choices are available in this template, the default is Linux Libertine, available for free at: http://www.linuxlibertine.org while the secondary choice is Hoefler Text which comes bundled with Mac OSX.
% To use Hoefler Text, comment out the Linux Libertine block below and uncomment the Hoefler Text block. You will also need to replace the "\&" characters with "\amper{}" in section titles.

% Linux Libertine Font (default)
%\setromanfont [Ligatures={Common}, Numbers={OldStyle}, Variant=01]{Linux Libertine O} % Main text font
\setromanfont{Linux Libertine}
\setmonofont[Scale=0.8]{Linux Libertine} % Set mono font (e.g. phone numbers)
\sectionfont{\mdseries\upshape\Large} % Set font options for sections
\subsectionfont{\mdseries\scshape\normalsize} % Set font options for subsections
\subsubsectionfont{\mdseries\upshape\large} % Set font options for subsubsections
\chardef\&="E050 % Custom ampersand character

% Hoefler Text Font (bundled with Mac OSX)
%\setromanfont [Ligatures={Common}, Numbers={OldStyle}]{Hoefler Text} % Main text font
%\setmonofont[Scale=0.8]{Monaco} % Set mono font (e.g. phone numbers)
%\setsansfont[Scale=0.9]{Optima Regular} % Set sans font, used in the main name and titles in the document
%\newcommand{\amper}{{\fontspec[Scale=.95]{Hoefler Text}\selectfont\itshape\&}} % Custom ampersand character
%\sectionfont{\sffamily\mdseries\large\underline} % Set font options for sections
%\subsectionfont{\rmfamily\mdseries\scshape\normalsize} % Set font options for subsections
%\subsubsectionfont{\rmfamily\bfseries\upshape\normalsize} % Set font options for subsubsections

\hyphenation{ма-те-риа-лы}
\hyphenation{диа-лек-то-ло-ги-чес-ких}
\hyphenation{ny-ky-ti-las-ta}
\hyphenation{na-ro-dov}
%----------------------------------------------------------------------------------------

\begin{document}

%----------------------------------------------------------------------------------------
%	CONTACT AND GENERAL INFORMATION SECTION
%----------------------------------------------------------------------------------------

{\LARGE Niko Tapio Partanen}\\[1cm] % Your name
Haapaniemenkatu 11 B 75\\
\texttt{00530} Helsinki\\[.2cm]
Puhelinnumero: \texttt{+358405068571}\\ % Your phone number
Sähköposti: \href{mailto:niko.partanen@kotus.fi}{niko.partanen@helsinki.fi}\\ % Your email address

%\vfill % Whitespace between contact information and specific CV information
\vspace{10mm}
%------------------------------------------------

Syntymäaika: 23. joulukuuta 1986 – Suomi\\ % Your date of birth

%------------------------------------------------

\section*{Työpaikka}

Tutkijakoulutettava, Helsingin yliopisto % Your current or previous employment position
%------------------------------------------------

\section*{Erikoistumisala}

Väitöskirjatutkimukseni käsittelee komisyrjäänin murteiden sosiolingvististä variaatiota. Työskentelen Koneen säätiön rahoittamassa tutkimushankkeessa `Language Technology meets Language Technology: The Next Step in the Description of Komi'.\\

\textbf{Avaintermit:} Uralilaiset kielet, sosiolingvistiikka, tietokonelingvistiikka, arkistoaineistot, kielten dokumentoinnin metodologia, tutkimusetiikka % Your primary areas of research interest

%----------------------------------------------------------------------------------------
%	WORK EXPERIENCE SECTION
%----------------------------------------------------------------------------------------

\section*{Aiemmat työtehtävät}

\years{2018-2019}Erityisasiantuntija, Kotimaisten kielten keskus\\
\years{2017-2018} Vieraileva tutkija, LATTICE-CNRS -laboratorio, Pariisi\\
\years{2016-2017}Ohjelmistokehittäjä, INEL / Hamburger Zentrum für Sprachkorpora\\
\years{2015}Ohjelmistokehittäjä, CLARIN / Hamburger Zentrum für Sprachkorpora\\
\years{2014-2015}Työntekijä projektissa: Izhva Komi Language Documentation, Koneen säätiö\\
\years{2013}Projektikoordinaattori hankkeessa Down River Vashka, Komi (Udoran murre), Koneen säätiö ja MinorEuRus\\
\years{2012-2013}Tutkimusavustaja, Helsingin yliopisto\\

%----------------------------------------------------------------------------------------
%	EDUCATION SECTION
%----------------------------------------------------------------------------------------

\pagebreak

\section*{Opinnot}

\years{2013}\textsc{MA} Suomalais-ugrilainen kielentutkimus, Helsingin yliopisto\\
\years{2013}\textsc{BA} Suomalais-ugrilainen kielentutkimus, Helsingin yliopisto\\
\years{syksy 2011}Vaihto-opiskelija, Syktyvkarin valtionyliopisto

%----------------------------------------------------------------------------------------
%	GRANTS, HONORS AND AWARDS SECTION
%----------------------------------------------------------------------------------------

\section*{Palkinnot}

\years{2014}Gradupalkinto, Venäjän ja Itä-Euroopan tutkimuksen seura

%----------------------------------------------------------------------------------------
%	PUBLICATIONS SECTION
%----------------------------------------------------------------------------------------

\section*{Julkaisut ja esitelmät}

\subsection*{Valitut artikkelit ja opinnäytteet}

\years{hyväksytty} KyungTae Lim, Niko Partanen \& Thierry Poibeau: Analyse syntaxique de langues faiblement dotées à partir de plongements de mots multilingues. Numéro spécial TAL.\\

\years{hyväksytty} Niko Partanen, Rogier Blokland, KyungTae Lim, Thierry Poibeau \& Michael Rießler: The First Komi-Zyrian Universal Dependencies Treebanks. Universal Dependencies Workshop 2018 (2018 Conference on Empirical Methods in Natural Language Processing), 1.11.2018, Brysseli, Belgia\\

\years{2018} Jeremy Bradley, Alexandra Kellner \& Niko Partanen: Variation in word order in Permic and Mari varieties – a corpus-based investigation. Языковые контакты народов Поволжья и Урала. XI Международный симпозиум (Чебоксары, 21–24 мая 2018 г.) – Сборник статей, Г. Е. Корнилов, А. М. Иванова, Г. Н. Семенова (toim.), s. 238–244, Chuvash University Publishing House, Cheboksary\\

\years{2018} KyungTae Lim, Niko Partanen \& Thierry Poibeau: Multilingual Dependency Parsing for Low-Resource Languages: Case Studies on North Saami and Komi-Zyrian. Proceedings of the 11th Language Resources and Evaluation Conference, 7.--12.5.2018, Miyazaki, Japani

\years{2018} Niko Partanen, KyungTae Lim, Michael Rießler \& Thierry Poibeau: Dependency Parsing of Code-Switching Data with Cross-Lingual Feature Representations. In Proceedings of the Fourth International Workshop on Computational Linguistics of Uralic Languages, s. 1--17\\

%\years{2017} Niko Partanen: Challenges in OCR today. Report on experiences from INEL. Konferenssijulkaisussa: Electronic Writing of RF Peoples: History, Issues, and Perspectives. 16.--17.3.2017, Syktyvkar\\

\years{2017} Ciprian Gerstenberger, Niko Partanen \& Michael Rießler: Instant annotations in ELAN corpora of spoken and written Komi, an endangered language of the Barents Sea region. Proceedings of the 2nd Workshop on the Use of Computational Methods in the Study of Endangered Languages, s. 57--66. \\

\years{2017} Ciprian Gerstenberger, Niko Partanen, Michael Rießler \& Joshua Wilbur: Instant Annotations -- Applying NLP Methods to the Annotation of Spoken Language Documentation Corpora. Proceedings of the Third Workshop on Computational Linguistics for Uralic Languages, s. 25--36.\\

\pagebreak

\years{2016} Ciprian Gerstenberger, Niko Partanen, Michael Rießler \& Joshua Wilbur: Utilizing language technology in the documentation of endangered Uralic languages. Northern European Journal of Language Technology, 4, 29-47 \\

\years{2016} Partanen, Niko \& Saarikivi, Janne (2016), Fragmentation of Karelian language and its community: growing variation at the threshold of the language shift. Teoksessa: \emph{Toivanen, Reetta \& Saarikivi, Janne (toim.), New and Old Language Diversities, Multilingual Matters}\\

%\years{2014} Rogier Blokland, Michael Rießler, Niko Partanen, Marina Fedina \& Andrei Chemyshev: Ispoľzovanie cifrovyx korpusov i komp’juternyx programm v dialektologičeskix issledovanijax: teoria i praktika. In: Khisamitdinova, F.G (ed). \emph{Räsäj xaliqtary teldäre dialektologijahynyŋ könüδäk mäśäläläre. XIV Bötä Räsäj fänni konferencija materialdary (Öfö, 20-22 nojabr’, 2014 j.). Aktuaľnye problemy dialektologii jazykov narodov Rossii: Materialy XIV Vserossijskoj naučnoj konferencii (Ufa, 20-22 nojabrja 2014 g.).} Ufa: IIJaL UNC RAN. 252–255.\\

\years{2013}Partanen, Niko (2013), Kahden suomalais-ugrilaisen yhteisön kielenvaihdon vertailevaa tarkastelua: Tuuksa Karjalassa ja Koigort Komissa, \emph{Julkaisematon pro gradu -tutkielma, Helsingin yliopisto}\\

\years{2013}Partanen, Niko (2013), Kielenvaihdon tarkastelua Vuokkiniemellä ja Viteleellä: Vuosien 2004 ja 2010 kenttätyöaineistojen analyysi, \emph{Julkaisematon kandidaatintutkielma, Helsingin yliopisto}\\


\subsection*{Valitut esitelmät}

\years{2017}Niko Partanen, Marina Fedina \& Michael Rießler: New perspectives on archiving practices in language documentation. 19.9., IASA 2017 -konferenssi, Berliini, Saksa\\

\years{2017}Niko Partanen: Issues with archiving linguistic data. Perspectives from work on Permic languages. Travaux de terrains anthropologiques et linguistiques : expériences et difficultés. 20.5., INALCO, Pariisi, Ranska\\

\years{2017}Ciprian Gerstenberger, Niko Partanen \& Michael Rießler: Instant annotations in ELAN corpora of spoken and written Komi-Zyrian, an endangered language of the Barents Sea region (Russia). ComputEL-2 (2nd Workshop on Computational Methods for Endangered Languages). 6.-7.3., Honolulu, Havaiji\\

%\years{2017} Niko Partanen: Challenges in OCR today. Report on experiences from INEL. Conference: Electronic Writing of RF Peoples: History, Issues, and Perspectives. 17.3., Syktyvkar\\

\years{2017} Niko Partanen: Russian influence across Komi registers. Colloquium "From Language Mixing to Fused Lects", Freiburg Institute for Advanced Studies, Albert-Ludwigs-Universität Freiburg, 25.--27.1., Freiburg, Saksa\\

%\years{2016} Daniel Jettka, Timm Lehmberg \& Niko Partanen: Digital workflows. INEL workshop. 03.11, INEL / Hamburger Zentrum für Sprachkorpora, Hampuri, Saksa.\\

\years{2016} Niko Partanen: Russian influence on Iźva Komi sibilant articulation. SLE conference, 25.8., Napoli, Italia\\

\years{2015} Hanna Thiele, Niko Partanen \& Michael Rießler: Exploring constituent order variation in selected languages of the Barents Sea area. Transalpine Typology Meeting, 8–9.10., Lyon, France\\

\pagebreak

\years{2015} Rogier Blokland, Michael Rießler, Niko Partanen \& Joshua Wilbur: A critical evaluation of past, current and future approaches in Uralic language documentation. XII International Congress for Finno-Ugric Studies, 17–20.8., Oulu, Suomi\\


\years{2015} Partanen, Niko \& Saarikivi, Janne: Linguistic variation in the Karelian communities. XII International Congress for Finno-Ugric Studies, 17–20.8., Oulu, Suomi\\

\years{2014} Rogier Blokland, Ciprian Gerstenberger, Marina Fedina, Niko Partanen, Michael Rießler, and Joshua Wilbur: Language documentation meets language technology. First International Workshop on Computational Linguistics for Uralic Languages. 16.1., Tromssa, Norja\\

\years{2013} Saarikivi, Janne \& Partanen, Niko: Rapid extinction or relative revitalization? Karelian and Komi language communities in comparison. International Conference on Minority Languages XIV, 11–14 September 2013, Graz, Itävalta\\ 

\years{2012}Partanen, Niko (2012), Poster presentation: Language shift in Komi, Sociolinguistics Symposium 19, Berliini, Saksa\\


%------------------------------------------------

%\subsection*{Kirja}

%\years{tulossa}Partanen, Niko \& Saarikivi, Janne (eds.) (2018), \emph{Materiaaleja karjalan kielen nykytilasta}

\section*{Opetus}

\years{2018} ELAN-tiedostojen käyttö korpustutkimuksessa. 14.–15.5. (Giellagas-instituutti)

\years{2017} Advanced analysis and manipulation of ELAN corpus data with R and Python, 16.--17.11. (Freiburg Institute for Advanced Studies)

\years{2017} Using ELAN to create and analyse spoken corpus, 1.--8.6. (Syktyvkarin valtionyliopisto)

\years{2015} ELAN-korpuksen käyttö tutkimuksessa, 19.--21.10. (Giellagas-instituutti)

%------------------------------------------------

%\subsection*{Taitot}

%\years{2017} Lüüdilaine 2017

%\years{2015} Lüüdilaine 2015

%------------------------------------------------

\section*{Kielitaito (käyttökielet)}

Suomi: äidinkieli\\
Komisyrjääni: erinomainen\\
Englanti: erinomainen\\
Italia: hyvä\\
Venäjä: hyvä\\
Udmurtti: hyvä keskustelutaito\\
Komipermjakki: hyvä keskustelutaito\\
Ruotsi: kehittyvä keskustelutaito\\
Saksa: perustiedot\\
Ranska: perustiedot\\

\section*{IT-taidot}

Hyvä osaaminen: R, Python, NLP-työkalut, LaTeX, XML, GIS-työkalut, OCR-työkalut, tietokannat, multimedian käsittely\\
Aktiivisesti käytössä: puheen- ja tekstintunnistuksen työkalut

\section*{Kenttätyökokemus}

\years{2016} Kuolan niemimaa, Murmanskin alue, kaksi viikkoa \emph{Rahoitus: Koneen Säätiö}\\ 
\years{2015} Eri alueet Udmurtiassa, kymmenen päivää \emph{Ei ulkoista rahoitusta}\\
\years{2015} Izhman piiri, Komin tasavalta, viisi päivää \emph{Rahoitus: Развития фундаментальных лингвистических исследований}\\
\years{2015} Narjan-Mar, Nenetsian autonominen okrug, kaksi viikkoa \emph{Rahoitus: Koneen Säätiö}\\
\years{2014} Izhman piiri, Komin tavalta, kaksi viikkoa \emph{Rahoitus: Koneen Säätiö}\\
\years{2013} Udoran piiri, Komin tasavalta, kuukausi \emph{Rahoitus: Koneen Säätiö ja MinorEuRus}\\
 \years{2013} Aunuksen piiri, Karjalan tasavalta, viikko \emph{Rahoitus: Helsingin yliopisto}\\
\years{2012} Koigorodokin piiri, Komin tasavalta, kaksi viikkoa \emph{Rahoitus: MinorEuRus}\\
 \years{2012} Aunuksen piiri, Karjalan tasavalta, kaksi viikkoa \emph{Rahoitus: Helsingin yliopisto}\\
 \years{2011} Prääsän piiri, Karjalan tasavalta, kaksi viikkoa \emph{Rahoitus: Helsingin yliopisto}\\
 \years{2010} Aunuksen piiri, Karjalan tasavalta, viikko \emph{Rahoitus: Helsingin yliopisto}


%----------------------------------------------------------------------------------------
%	FINAL FOOTER
%----------------------------------------------------------------------------------------

%\begin{center}
%{\scriptsize Last updated: \today}
%\end{center}

%----------------------------------------------------------------------------------------

\end{document}