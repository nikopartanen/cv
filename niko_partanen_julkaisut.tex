% !TEX encoding = UTF-8 Unicode

%----------------------------------------------------------------------------------------
%	PACKAGES AND OTHER DOCUMENT CONFIGURATIONS
%----------------------------------------------------------------------------------------

\documentclass[11pt, a4paper]{article}

\usepackage{xltxtra,fontspec,xunicode,graphicx,graphics,geometry,setspace,multicol,multirow}


\geometry{a4paper, textwidth=5.5in, textheight=8.5in, marginparsep=7pt, marginparwidth=.6in}
\setlength\parindent{0in}

\usepackage[usenames,dvipsnames]{xcolor}

\usepackage{linguex}
\usepackage{sectsty} 

\usepackage[normalem]{ulem} 
\usepackage{xunicode} 

\defaultfontfeatures{Mapping=tex-text} 

\usepackage{marginnote} % For margin years
\newcommand{\years}[1]{\marginnote{\scriptsize #1}} % New command for including margin years
\renewcommand*{\raggedleftmarginnote}{}
\setlength{\marginparsep}{7pt} % Slightly increase the distance of the margin years from the contant
\reversemarginpar

\usepackage[xetex, bookmarks, colorlinks, breaklinks, pdftitle={Niko Partanen - curriculum vitae},pdfauthor={Niko Partanen}]{hyperref} 

\hypersetup{linkcolor=blue,citecolor=blue,filecolor=black,urlcolor=MidnightBlue} 

%----------------------------------------------------------------------------------------
%	FONT CONFIGURATIONS
%----------------------------------------------------------------------------------------

% Two font choices are available in this template, the default is Linux Libertine, available for free at: http://www.linuxlibertine.org while the secondary choice is Hoefler Text which comes bundled with Mac OSX.
% To use Hoefler Text, comment out the Linux Libertine block below and uncomment the Hoefler Text block. You will also need to replace the "\&" characters with "\amper{}" in section titles.

% Linux Libertine Font (default)
%\setromanfont [Ligatures={Common}, Numbers={OldStyle}, Variant=01]{Linux Libertine O} % Main text font
\setromanfont{Linux Libertine}
\setmonofont[Scale=0.8]{Linux Libertine} % Set mono font (e.g. phone numbers)
\sectionfont{\mdseries\upshape\Large} % Set font options for sections
\subsectionfont{\mdseries\scshape\normalsize} % Set font options for subsections
\subsubsectionfont{\mdseries\upshape\large} % Set font options for subsubsections
\chardef\&="E050 % Custom ampersand character

% Hoefler Text Font (bundled with Mac OSX)
%\setromanfont [Ligatures={Common}, Numbers={OldStyle}]{Hoefler Text} % Main text font
%\setmonofont[Scale=0.8]{Monaco} % Set mono font (e.g. phone numbers)
%\setsansfont[Scale=0.9]{Optima Regular} % Set sans font, used in the main name and titles in the document
%\newcommand{\amper}{{\fontspec[Scale=.95]{Hoefler Text}\selectfont\itshape\&}} % Custom ampersand character
%\sectionfont{\sffamily\mdseries\large\underline} % Set font options for sections
%\subsectionfont{\rmfamily\mdseries\scshape\normalsize} % Set font options for subsections
%\subsubsectionfont{\rmfamily\bfseries\upshape\normalsize} % Set font options for subsubsections

\hyphenation{ма-те-риа-лы}
\hyphenation{диа-лек-то-ло-ги-чес-ких}
\hyphenation{ny-ky-ti-las-ta}
\hyphenation{na-ro-dov}
\hyphenation{кон-фе-рен-ции}
%----------------------------------------------------------------------------------------

\begin{document}

%----------------------------------------------------------------------------------------
%	CONTACT AND GENERAL INFORMATION SECTION
%----------------------------------------------------------------------------------------

{\LARGE Niko Tapio Partanen}\\[1cm] % Your name
Lattice-laboratorio\\
CNRS \& Ecole normale supérieure/PSL \& Univ. Sorbonne nouvelle / USPC\\
1 rue Maurice Arnoux \texttt{F-92120}\\
Montrouge Ranska\\[.2cm]
Puhelinnumero: \texttt{+358405068571}\\ % Your phone number
Sähköposti: \href{mailto:nikotapiopartanen@gmail.com}{nikotapiopartanen@gmail.com}\\ % Your email address

%----------------------------------------------------------------------------------------
%	PUBLICATIONS SECTION
%----------------------------------------------------------------------------------------

Tämä julkaisuluettelo sisältää tieteelliset julkaisuni, esitelmäni ja opetukseni. Se ei sisällä julkaisujen taittoon tai kielentarkistukseen liittyviä töitäni, eikä kenttätyökokemustani.

\section*{Julkaisuluettelo}

\subsection*{Artikkelit}

\years{hyväksytty} KyungTae Lim, Niko Partanen \& Thierry Poibeau: Multilingual Dependency Parsing for Low-Resource Languages: Case Studies on North Saami and Komi-Zyrian. Proceedings of the 11th Language Resources and Evaluation Conference, 7.--12.5.2018, Miyazaki, Japani\\

\years{2018} Niko Partanen, KyungTae Lim, Michael Rießler \& Thierry Poibeau: Dependency Parsing of Code-Switching Data with Cross-Lingual Feature Representations. Proceedings of the Fourth International Workshop on Computational Linguistics of Uralic Languages, s. 1--17. URL: \url{http://www.aclweb.org/anthology/W18-0201}\\

\years{2017} Niko Partanen: Challenges in OCR today. Report on experiences from INEL. Electronic Writing of RF Peoples: History, Issues, and Perspectives. 16.--17.3.2017, Syktyvkar, s. 263--273. URL: \url{http://fu-lab.ru/conference/sites/all/themes/conference/files}\\

\years{2017} Ciprian Gerstenberger, Niko Partanen \& Michael Rießler: Instant annotations in ELAN corpora of spoken and written Komi, an endangered language of the Barents Sea region. Proceedings of the 2nd Workshop on the Use of Computational Methods in the Study of Endangered Languages, s. 57--66. URL: \url{http://www.aclweb.org/anthology/W17-0109}\\

\years{2017} Ciprian Gerstenberger, Niko Partanen, Michael Rießler \& Joshua Wilbur: Instant Annotations -- Applying NLP Methods to the Annotation of Spoken Language Documentation Corpora. Proceedings of the Third Workshop on Computational Linguistics for Uralic Languages, s. 25--36. URL: \url{http://www.aclweb.org/anthology/W17-0604}\\

\years{2016} Ciprian Gerstenberger, Niko Partanen, Michael Rießler \& Joshua Wilbur: Utilizing language technology in the documentation of endangered Uralic languages. Northern European Journal of Language Technology, 4, s. 29--47. URL: \url{http://www.nejlt.ep.liu.se/2016/v4/a03/nejlt16v4a3.pdf} \\

\years{2016} Partanen, Niko \& Saarikivi, Janne: Fragmentation of Karelian language and its community: growing variation at the threshold of the language shift. Teoksessa Toivanen, Reetta \& Saarikivi, Janne (toim.): \emph{New and Old Language Diversities}, Multilingual Matters\\

\years{2014} Rogier Blokland, Michael Rießler, Niko Partanen, Marina Fedina \& Andrei Chemyshev: Использование цифровых корпусов и компьютерных программ в диалектологических исследованиях: теория и практика. Teoksessa Khisamitdinova, F.G (toim.): \emph{Рәсәй халыҡтары телдәре диалектологияһының көнүҙәк мәсьәләләре. XIV Бөтә Рәсәй фәнни конференция материалдары (Өфө, 20--22 ноябрь, 2014 й.). Актуальные проблемы диалектологии языков народов России: Материалы XIV Всероссийской научной конференции (Уфа, 20--22 ноябра 2014 г.).} Уфа: ИИЯЛ УНЦ РАН. s. 252--255.\\

\subsection*{Julkaisemattomat teokset}

\years{2013}Partanen, Niko: Kahden suomalais-ugrilaisen yhteisön kielenvaihdon vertailevaa tarkastelua: Tuuksa Karjalassa ja Koigort Komissa. Julkaisematon pro gradu -tutkielma, Helsingin yliopisto. URN: \url{http://urn.fi/URN:NBN:fi-fe201401221241}\\

\years{2013}Partanen, Niko: Kielenvaihdon tarkastelua Vuokkiniemellä ja Viteleellä: Vuosien 2004 ja 2010 kenttätyöaineistojen analyysi. Julkaisematon kandidaatintutkielma, Helsingin yliopisto.\\


%------------------------------------------------

\subsection*{Kirja}

\years{tulossa}Partanen, Niko \& Saarikivi, Janne (eds.): \emph{Materiaaleja karjalan kielen nykytilasta}

\subsection*{Esitelmät}

\years{2017}Niko Partanen, Marina Fedina \& Michael Rießler: New perspectives on archiving practices in language documentation. 19.9., IASA 2017 -konferenssi, Berliini, Saksa\\

\years{2017}Niko Partanen: Issues with archiving linguistic data. Perspectives from work on Permic languages. Travaux de terrains anthropologiques et linguistiques : expériences et difficultés, 20.5., INALCO, Pariisi, Ranska\\

\years{2017}Ciprian Gerstenberger, Niko Partanen \& Michael Rießler: Instant annotations in ELAN corpora of spoken and written Komi-Zyrian, an endangered language of the Barents Sea region (Russia). ComputEL-2 (2nd Workshop on Computational Methods for Endangered Languages), 6.–7.3., Honolulu, Yhdysvallat\\

\newpage

\years{2017} Niko Partanen: Challenges in OCR today. Report on experiences from INEL. Konferenssi: Electronic Writing of RF Peoples: History, Issues, and Perspectives, 17.3., Syktyvkar\\

\years{2017} Niko Partanen: Russian influence across Komi registers. Colloquium "From Language Mixing to Fused Lects", Freiburg Institute for Advanced Studies, Albert-Ludwigs-Universität Freiburg, 25.–27.1., Freiburg, Saksa\\

\years{2016} Daniel Jettka, Timm Lehmberg \& Niko Partanen: Digital workflows. INEL työpaja, 03.11, INEL / Hamburger Zentrum für Sprachkorpora, Hampuri, Saksa.\\

\years{2016} Niko Partanen: Russian influence on Iźva Komi sibilant articulation. SLE-konferenssi, 25.8., Napoli, Italia\\

\years{2015} Hanna Thiele, Niko Partanen \& Michael Rießler: Exploring constituent order variation in selected languages of the Barents Sea area. Transalpine Typology Meeting, 8.–9.10., Lyon, Ranska\\

\years{2015} Rogier Blokland, Michael Rießler, Niko Partanen \& Joshua Wilbur: A critical evaluation of past, current and future approaches in Uralic language documentation. XII International Congress for Finno-Ugric Studies, 17.–20.8., Oulu, Suomi\\

\years{2015} Partanen, Niko \& Saarikivi, Janne: Linguistic variation in the Karelian communities. XII International Congress for Finno-Ugric Studies, 17–20.8., Oulu, Suomi\\

\years{2014} Rogier Blokland, Ciprian Gerstenberger, Marina Fedina, Niko Partanen, Michael Rießler, \& Joshua Wilbur: Language documentation meets language technology. First International Workshop on Computational Linguistics for Uralic Languages, 16.1., Tromssa, Norja\\

\years{2013} Saarikivi, Janne \& Partanen, Niko: Rapid extinction or relative revitalization? Karelian and Komi language communities in comparison. International Conference on Minority Languages XIV, 11.–.14.9., Graz, Itävalta\\ 

\years{2012}Partanen, Niko: Language shift in Komi (juliste). Sociolinguistics Symposium 19, 21.--24.8., Berliini, Saksa\\

\subsection*{Opetus}

\years{2017} Advanced analysis and manipulation of ELAN corpus data with R and Python -kurssi. 16.--17.11. (Freiburg Institute for Advanced Studies)

\years{2017} Using ELAN to create and analyse spoken corpus -kurssi. 1.--8.6. (Syktyvkarin valtionyliopisto)

\years{2015} ELAN-korpuksen käyttö tutkimuksessa -työpaja. 19.--21.10. (Giellagas-instituutti)

\end{document}