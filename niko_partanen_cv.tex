% !TEX encoding = UTF-8 Unicode

%----------------------------------------------------------------------------------------
%	PACKAGES AND OTHER DOCUMENT CONFIGURATIONS
%----------------------------------------------------------------------------------------

\documentclass[11pt, a4paper]{article}

\usepackage{xltxtra,fontspec,xunicode,graphicx,graphics,geometry,setspace,multicol,multirow}


\geometry{a4paper, textwidth=5.5in, textheight=8.8in, marginparsep=7pt, marginparwidth=.6in}
\setlength\parindent{0in}

\usepackage[usenames,dvipsnames]{xcolor}

\usepackage{linguex}
\usepackage{sectsty} 

\usepackage[normalem]{ulem} 
\usepackage{xunicode} 

\defaultfontfeatures{Mapping=tex-text} 

\usepackage{marginnote} % For margin years
\newcommand{\years}[1]{\marginnote{\scriptsize #1}} % New command for including margin years
\renewcommand*{\raggedleftmarginnote}{}
\setlength{\marginparsep}{7pt} % Slightly increase the distance of the margin years from the contant
\reversemarginpar

\usepackage[xetex, bookmarks, colorlinks, breaklinks, pdftitle={Niko Partanen - curriculum vitae},pdfauthor={Niko Partanen}]{hyperref} 

\hypersetup{linkcolor=blue,citecolor=blue,filecolor=black,urlcolor=MidnightBlue} 

%----------------------------------------------------------------------------------------
%	FONT CONFIGURATIONS
%----------------------------------------------------------------------------------------

% Two font choices are available in this template, the default is Linux Libertine, available for free at: http://www.linuxlibertine.org while the secondary choice is Hoefler Text which comes bundled with Mac OSX.
% To use Hoefler Text, comment out the Linux Libertine block below and uncomment the Hoefler Text block. You will also need to replace the "\&" characters with "\amper{}" in section titles.

% Linux Libertine Font (default)
%\setromanfont [Ligatures={Common}, Numbers={OldStyle}, Variant=01]{Linux Libertine O} % Main text font
\setromanfont{Linux Libertine}
\setmonofont[Scale=0.8]{Linux Libertine} % Set mono font (e.g. phone numbers)
\sectionfont{\mdseries\upshape\Large} % Set font options for sections
\subsectionfont{\mdseries\scshape\normalsize} % Set font options for subsections
\subsubsectionfont{\mdseries\upshape\large} % Set font options for subsubsections
\chardef\&="E050 % Custom ampersand character

% Hoefler Text Font (bundled with Mac OSX)
%\setromanfont [Ligatures={Common}, Numbers={OldStyle}]{Hoefler Text} % Main text font
%\setmonofont[Scale=0.8]{Monaco} % Set mono font (e.g. phone numbers)
%\setsansfont[Scale=0.9]{Optima Regular} % Set sans font, used in the main name and titles in the document
%\newcommand{\amper}{{\fontspec[Scale=.95]{Hoefler Text}\selectfont\itshape\&}} % Custom ampersand character
%\sectionfont{\sffamily\mdseries\large\underline} % Set font options for sections
%\subsectionfont{\rmfamily\mdseries\scshape\normalsize} % Set font options for subsections
%\subsubsectionfont{\rmfamily\bfseries\upshape\normalsize} % Set font options for subsubsections

\hyphenation{ма-те-риа-лы}
\hyphenation{диа-лек-то-ло-ги-чес-ких}
\hyphenation{ny-ky-ti-las-ta}
\hyphenation{na-ro-dov}
%----------------------------------------------------------------------------------------

\begin{document}

%----------------------------------------------------------------------------------------
%	REMOVES THE PAGE NUMBER
%----------------------------------------------------------------------------------------

\thispagestyle{empty}

%----------------------------------------------------------------------------------------
%	CONTACT AND GENERAL INFORMATION SECTION
%----------------------------------------------------------------------------------------

{\LARGE Niko Tapio Partanen}\\[1cm] % Your name
Haapaniemenkatu 11 B 75\\
\texttt{00530} Helsinki\\
Finland\\[.2cm]
Phone: \texttt{+358405068571}\\ % Your phone number
Email: \href{mailto:nikotapiopartanen@gmail.com}{nikotapiopartanen@gmail.com}\\ % Your email address
% Blog: \href{http://langdoc.github.io}{langdoc.github.io (multiple authors)}\\ % Your academic/personal website

% \medbreak
% \vfill % Whitespace between contact information and specific CV information

%------------------------------------------------

Born: December 23, 1986–Finland\\ % Your date of birth
%Nationality: Finnish % Your nationality

%------------------------------------------------

\section*{Current position}

Senior Specialist, Institute for the Languages of Finland (until the end of May 2019).

\vspace{3mm}

I'm conducting my PhD studies in the University in Helsinki starting in summer 2019, the estimated year of graduation being 2021. My first supervisor is Michael Rießler. The work is conducted within the research project \textbf{Language Documentation meets Language Technology: The Next Step in the Description of Komi}.

% Your current or previous employment position
%------------------------------------------------

\section*{Areas of specialization}

I study sociolinguistic variation in the Permic languages and investigate the applicability of computational methods to documentary linguistics data. My earlier research has also focused on linguistic endangerment and revitalization. 

%\vspace{3mm}

%I have worked extensively with archiving and data management practices, with a focus on information retrieval from documents and storage of metadata about archive items.\\ 

% \textbf{Keywords:} Sociolinguistics, endangered languages, documentary linguistics, Natural\\ Language Processing, corpus linguistics, research methodology, Zyrian Komi % Your primary areas of research interest

%----------------------------------------------------------------------------------------
%	WORK EXPERIENCE SECTION
%----------------------------------------------------------------------------------------

\section*{Positions held \& Grants received}

\years{2017-2018} Visiting researcher, Lattice laboratory\\
\years{2017-2021}Co-applicant in project \textbf{Language Documentation meets Language Technology: The Next Step in the Description of Komi}, funded by Kone Foundation.\\ % Remaining personal grant: 24 months. \\
\years{2015-2017}Software developer, INEL / Hamburger Zentrum für Sprachkorpora\\
\years{2014-2015}PhD student in project \textbf{Izhva Komi Language Documentation}, funded by  Kone Foundation. Freiburg Research Group in Saami Studies. Used personal grant: 22 months. \\
\years{2013}PI in project \textbf{Down River Vashka}, Komi (Udora dialect), funded by Kone Foundation and MinorEuRus. Used personal grant: 1 month.\\
% \years{2012-2013}Research assistant, University of Helsinki\\

%----------------------------------------------------------------------------------------
%	EDUCATION SECTION
%----------------------------------------------------------------------------------------

\section*{Education}

\years{2013}\textsc{MA} in Finno-Ugric linguistics, University of Helsinki\\
\years{2013}\textsc{BA} in Finno-Ugric linguistics, University of Helsinki\\
\years{fall 2011}north2north program exchange student, Syktyvkar State University

%----------------------------------------------------------------------------------------
%	GRANTS, HONORS AND AWARDS SECTION
%----------------------------------------------------------------------------------------

\section*{Grants, honors \& awards}

\years{2014}VIETS Master Thesis Prize, Finnish Association for Russian and Eastern European Studies

%----------------------------------------------------------------------------------------
%	PUBLICATIONS SECTION
%----------------------------------------------------------------------------------------

%\section*{Publications and Presentations}

\section*{Journal articles and final papers}

\years{2019}Rogier Blokland, Niko Partanen, Michael Rießler \& Joshua Wilbur: Using computational approaches to integrate endangered language legacy data into documentation corpora: past experiences and challenges ahead. In: Proceedings of the Workshop on Computational Methods for Endangered Languages, Honolulu, Hawai’i, USA, February 26–27, 2019. Vol. 2/5: pp. 24-30. URL: 
\url{https://scholar.colorado.edu/scil-cmel/vol2/iss1/5}\\

\years{2019}Partanen, Niko \& Michael Rießler: An OCR system for the Unified Northern Alphabet. Proceedings of the Fifth International Workshop on Computational Linguistics for Uralic Languages. 2019. URL: \url{https://www.aclweb.org/anthology/W19-0307}\\

\years{2018}KyungTae Lim, Niko Partanen \& Thierry Poibeau: Analyse syntaxique de langues faiblement dotées à partir de plongements de mots multilingues. Application au same du nord et au komi-zyriène. TAL. Volume 59 – n◦ 3/2018, pp. 67-91\\

\years{2018} Niko Partanen, Rogier Blokland, Kyungtae Lim, Thierry Poibeau \& Michael Rießler: The First Komi- Zyrian Universal Dependencies Treebanks. Universal dependencies Workshop, Brussels, Belgium, 2018: 126-132. URL: \url{https://www.aclweb.org/anthology/W18-60#page=140}\\

\years{2018} KyungTae Lim, Niko Partanen \& Thierry Poibeau: Multilingual Dependency Parsing for Low-Resource Languages: Case Studies on North Saami and Komi-Zyrian. Proceedings of the 11th Language Resources and Evaluation Conference, 7.--12.5.2018, Miyazaki, Japani. URL: \url{http://www.lrec-conf.org/proceedings/lrec2018/pdf/600.pdf}\\

\years{2018} Niko Partanen, KyungTae Lim, Michael Rießler \& Thierry Poibeau: Dependency Parsing of Code-Switching Data with Cross-Lingual Feature Representations. Proceedings of the Fourth International Workshop on Computational Linguistics of Uralic Languages, s. 1--17. URL: \url{http://www.aclweb.org/anthology/W18-0201}\\

\years{2018} Jeremy Bradley, Alexandra Kellner \& Niko Partanen: Variation in word order in Permic and Mari varieties – a corpus-based investigation. Языковые контакты народов Поволжья и Урала. XI Международный симпозиум (Чебоксары, 21–24 мая 2018 г.) – Сборник статей, Г. Е. Корнилов, А. М. Иванова, Г. Н. Семенова (toim.), s. 238–244, Chuvash University Publishing House, Cheboksary\\

\years{2017} Niko Partanen: Challenges in OCR today. Report on experiences from INEL. Electronic Writing of RF Peoples: History, Issues, and Perspectives. 16.--17.3.2017, Syktyvkar, s. 263--273. URL: \url{http://fu-lab.ru/conference/sites/all/themes/conference/files}\\

\years{2017} Ciprian Gerstenberger, Niko Partanen \& Michael Rießler: Instant annotations in ELAN corpora of spoken and written Komi, an endangered language of the Barents Sea region. Proceedings of the 2nd Workshop on the Use of Computational Methods in the Study of Endangered Languages, s. 57--66. URL: \url{http://www.aclweb.org/anthology/W17-0109}\\

\years{2017} Ciprian Gerstenberger, Niko Partanen, Michael Rießler \& Joshua Wilbur: Instant Annotations -- Applying NLP Methods to the Annotation of Spoken Language Documentation Corpora. Proceedings of the Third Workshop on Computational Linguistics for Uralic Languages, s. 25--36. URL: \url{http://www.aclweb.org/anthology/W17-0604}\\

\years{2016} Ciprian Gerstenberger, Niko Partanen, Michael Rießler \& Joshua Wilbur: Utilizing language technology in the documentation of endangered Uralic languages. Northern European Journal of Language Technology, 4, 29-47 \\

\years{2016} Partanen, Niko \& Saarikivi, Janne (2016), Fragmentation of Karelian language and its community: growing variation at the threshold of the language shift. In: \emph{Toivanen, Reetta \& Saarikivi, Janne (eds.), New and Old Language Diversities, Multilingual Matters}\\

\years{2014} Rogier Blokland, Michael Rießler, Niko Partanen, Marina Fedina \& Andrei Chemyshev: Ispoľzovanie cifrovyx korpusov i komp’juternyx programm v dialektologičeskix issledovanijax: teoria i praktika. In: Khisamitdinova, F.G (ed). \emph{Räsäj xaliqtary teldäre dialektologijahynyŋ könüδäk mäśäläläre. XIV Bötä Räsäj fänni konferencija materialdary (Öfö, 20-22 nojabr’, 2014 j.). Aktuaľnye problemy dialektologii jazykov narodov Rossii: Materialy XIV Vserossijskoj naučnoj konferencii (Ufa, 20-22 nojabrja 2014 g.).} Ufa: IIJaL UNC RAN. 252–255.\\

\years{2013}Partanen, Niko (2013), Kahden suomalais-ugrilaisen yhteisön kielenvaihdon vertailevaa tarkastelua: Tuuksa Karjalassa ja Koigort Komissa (Comparative examination of the language shift in two Uralic speech communities: Tuuksa in Karelia and Koigort in Komi), \emph{Unpublished Master Thesis, University of Helsinki}\\

\years{2013}Partanen, Niko (2013), Kielenvaihdon tarkastelua Vuokkiniemellä ja Viteleellä: Vuosien 2004 ja 2010 kenttätyöaineistojen analyysi (Observating the language shift in Vuokkiniemi and Vitele: analysis of the fieldwork materials from 2004 and 2010), \emph{Unpublished Bachelor Thesis, University of Helsinki}\\

%\subsection*{Presentations}

%\years{2017}Niko Partanen: Issues with archiving linguistic data. Perspectives from work on Permic languages. Travaux de terrains anthropologiques et linguistiques : expériences et difficultés. 20, May, INALCO, Paris, France\\

%\years{2017}Ciprian Gerstenberger, Niko Partanen \& Michael Rießler: Instant annotations in ELAN corpora of spoken and written Komi-Zyrian, an endangered language of the Barents Sea region (Russia). ComputEL-2 (2nd Workshop on Computational Methods for Endangered Languages). 6-7 March, 2017, Honolulu, Hawaii\\

%\years{2017} Niko Partanen: Challenges in OCR today. Report on experiences from INEL. Conference: Electronic Writing of RF Peoples: History, Issues, and Perspectives. 17 March, 2017, Syktyvkar\\

%\years{2017} Niko Partanen: Russian influence across Komi registers. Colloquium "From Language Mixing to Fused Lects", Freiburg Institute for Advanced Studies, Albert-Ludwigs-Universität Freiburg, 25.--27. January, 2017, Freiburg\\

%\years{2016} Daniel Jettka, Timm Lehmberg \& Niko Partanen: Digital workflows. INEL workshop. 03. November, 2016. INEL / Hamburger Zentrum für Sprachkorpora, Hamburg.\\

%\years{2016} Niko Partanen: Russian influence on Iźva Komi sibilant articulation. SLE conference, 25. August, 2016, Naples, Italy\\

%\years{2015} Hanna Thiele, Niko Partanen \& Michael Rießler: Exploring constituent order variation in selected languages of the Barents Sea area. Transalpine Typology Meeting, 8–9 October, 2015, Lyon, France\\

%\years{2015} Rogier Blokland, Michael Rießler, Niko Partanen \& Joshua Wilbur: A critical evaluation of past, current and future approaches in Uralic language documentation. XII International Congress for Finno-Ugric Studies, 17–20 August, 2015, Oulu, Finland\\

%\years{2015} Partanen, Niko \& Saarikivi, Janne: Linguistic variation in the Karelian communities. XII International Congress for Finno-Ugric Studies, 17–20 August, 2015, Oulu, Finland\\

%\years{2014} Rogier Blokland, Ciprian Gerstenberger, Marina Fedina, Niko Partanen, Michael Rießler, and Joshua Wilbur: Language documentation meets language technology. First International Workshop on Computational Linguistics for Uralic Languages. 16th January, 2015, Tromsø, Norway\\

%\years{2013} Saarikivi, Janne \& Partanen, Niko: Rapid extinction or relative revitalization? Karelian and Komi language communities in comparison. International Conference on Minority Languages XIV, 11–14 September 2013, Graz, Austria\\ 

%\years{2012}Partanen, Niko: Poster presentation: Language shift in Komi, Sociolinguistics Symposium 19, Berlin\\


%------------------------------------------------

% \subsection*{Books}

% \years{forthcoming}Partanen, Niko \& Saarikivi, Janne (eds.) (2017), \emph{Materiaaleja karjalan kielen nykytilasta (Materials on the current state of the Karelian language)}

\subsection*{Teaching}

\years{2017} Using ELAN to create and analyse spoken corpus, June 1--8. (Syktyvkar State University)

%------------------------------------------------

\subsection*{Language skills}

Finnish: native\\
Zyrian Komi: very good\\
English: very good\\
Italian: good\\
Russian: good\\
German: conversational\\
Swedish: conversational\\
Udmurt: conversational\\
Mandarin Chinese: beginner\\
French: beginner\\

\subsection*{Computer skills}

\textbf{Good knowledge:} R, Python, NLP tools, LaTeX, XML, GIS tools, OCR tools\\
\textbf{Learning / actively using:} Keras, TensorFlow, audio and image processing methods

\subsection*{Recent fieldwork experience with Permic languages}

\years{2016} Kola Peninsula, Murmanskiy Oblast, two weeks \emph{Funded by Kone Foundation}\\ 
\years{2015} Different regions in Udmurtia, ten days \emph{No external funding}\\
\years{2015} Izhemsky rayon, Komi Republic, five days \emph{Funded by Развития фундаментальных лингвистических исследований}\\
\years{2015} Naryan-Mar, Nenetskiy Autonomous Okrug, two weeks \emph{Funded by Kone Foundation}\\
\years{2014} Izhemsky rayon, Komi Republic, two weeks \emph{Funded by Kone Foundation}\\
\years{2013} Udorsky rayon, Komi Republic, a month \emph{Funded by Kone Foundation and MinorEuRus}\\
 \years{2013} Olonetsky rayon, Republic of Karelia, a week \emph{Funded by University of Helsinki}\\
\years{2012} Koygorodsky rayon, Komi Republic, two weeks \emph{Funded by MinorEuRus}\\
 \years{2012} Olonetsky rayon, Republic of Karelia, two weeks \emph{Funded by University of Helsinki}\\
\years{2011} Pryazhinsky rayon, Republic of Karelia, two weeks \emph{Funded by University of Helsinki}\\
\years{2010} Olonetsky rayon, Republic of Karelia, a week \emph{Funded by University of Helsinki}

%----------------------------------------------------------------------------------------
%	FINAL FOOTER
%----------------------------------------------------------------------------------------

%\begin{center}
%{\scriptsize Last updated: \today}
%\end{center}

%----------------------------------------------------------------------------------------

\end{document}