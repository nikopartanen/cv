% !TEX encoding = UTF-8 Unicode

%----------------------------------------------------------------------------------------
%	PACKAGES AND OTHER DOCUMENT CONFIGURATIONS
%----------------------------------------------------------------------------------------

\documentclass[11pt, a4paper]{article}

\usepackage{xltxtra,fontspec,xunicode,graphicx,graphics,geometry,setspace,multicol,multirow}


\geometry{a4paper, textwidth=5.5in, textheight=8.5in, marginparsep=7pt, marginparwidth=.6in}
\setlength\parindent{0in}

\usepackage[usenames,dvipsnames]{xcolor}

\usepackage{linguex}
\usepackage{sectsty} 

\usepackage[normalem]{ulem} 
\usepackage{xunicode} 

\defaultfontfeatures{Mapping=tex-text} 

\usepackage{marginnote} % For margin years
\newcommand{\years}[1]{\marginnote{\scriptsize #1}} % New command for including margin years
\renewcommand*{\raggedleftmarginnote}{}
\setlength{\marginparsep}{7pt} % Slightly increase the distance of the margin years from the contant
\reversemarginpar

\usepackage[xetex, bookmarks, colorlinks, breaklinks, pdftitle={Niko Partanen - curriculum vitae},pdfauthor={Niko Partanen}]{hyperref} 

\hypersetup{linkcolor=blue,citecolor=blue,filecolor=black,urlcolor=MidnightBlue} 

%----------------------------------------------------------------------------------------
%	FONT CONFIGURATIONS
%----------------------------------------------------------------------------------------

% Two font choices are available in this template, the default is Linux Libertine, available for free at: http://www.linuxlibertine.org while the secondary choice is Hoefler Text which comes bundled with Mac OSX.
% To use Hoefler Text, comment out the Linux Libertine block below and uncomment the Hoefler Text block. You will also need to replace the "\&" characters with "\amper{}" in section titles.

% Linux Libertine Font (default)
%\setromanfont [Ligatures={Common}, Numbers={OldStyle}, Variant=01]{Linux Libertine O} % Main text font
\setromanfont{Linux Libertine}
\setmonofont[Scale=0.8]{Linux Libertine} % Set mono font (e.g. phone numbers)
\sectionfont{\mdseries\upshape\Large} % Set font options for sections
\subsectionfont{\mdseries\scshape\normalsize} % Set font options for subsections
\subsubsectionfont{\mdseries\upshape\large} % Set font options for subsubsections
\chardef\&="E050 % Custom ampersand character

% Hoefler Text Font (bundled with Mac OSX)
%\setromanfont [Ligatures={Common}, Numbers={OldStyle}]{Hoefler Text} % Main text font
%\setmonofont[Scale=0.8]{Monaco} % Set mono font (e.g. phone numbers)
%\setsansfont[Scale=0.9]{Optima Regular} % Set sans font, used in the main name and titles in the document
%\newcommand{\amper}{{\fontspec[Scale=.95]{Hoefler Text}\selectfont\itshape\&}} % Custom ampersand character
%\sectionfont{\sffamily\mdseries\large\underline} % Set font options for sections
%\subsectionfont{\rmfamily\mdseries\scshape\normalsize} % Set font options for subsections
%\subsubsectionfont{\rmfamily\bfseries\upshape\normalsize} % Set font options for subsubsections

\hyphenation{ма-те-риа-лы}
\hyphenation{диа-лек-то-ло-ги-чес-ких}
\hyphenation{ny-ky-ti-las-ta}
\hyphenation{na-ro-dov}
%----------------------------------------------------------------------------------------

\begin{document}

%----------------------------------------------------------------------------------------
%	CONTACT AND GENERAL INFORMATION SECTION
%----------------------------------------------------------------------------------------

{\LARGE Niko Tapio Partanen}\\[1cm] % Your name
Lattice-CNRS\\ % Your address
Ecole Normale Supérieure\\
1 rue Maurice Arnoux \texttt{F-92120}\\
Montrouge France\\[.2cm]
Phone: \texttt{+358405068571}\\ % Your phone number
Email: \href{mailto:nikotapiopartanen@gmail.com}{nikotapiopartanen@gmail.com}\\ % Your email address
% \textsc{url}: \href{http://...}\\ % Your academic/personal website

\vfill % Whitespace between contact information and specific CV information

%------------------------------------------------

Born: December 23, 1986–Finland\\ % Your date of birth
Nationality: Finnish % Your nationality

%------------------------------------------------

\section*{Current position}

Researcher, Lattice-CNRS / Ecole Normale Supérieure (Apr 2017–present)% Your current or previous employment position
%------------------------------------------------

\section*{Areas of specialization}

I study sociolinguistic variation in Permic languages, but as a continuous side-project I investigate the applicability of computational methods to documentary linguistics data.

\textbf{Keywords:} Sociolinguistics, endangered languages, documentary linguistics, Natural\\ Language Processing, corpus linguistics, Permic languages, research methodology % Your primary areas of research interest

%----------------------------------------------------------------------------------------
%	WORK EXPERIENCE SECTION
%----------------------------------------------------------------------------------------

\section*{Positions held}

\years{2017-2019}PI in project \textbf{Language Documentation meets Language Technology: The Next Step in the Description of Komi}, funded by Kone Foundation\\
\years{2015-2017}Software developer, INEL / Hamburger Zentrum für Sprachkorpora\\
\years{2014-2015}PhD student in project \textbf{Izhva Komi Language Documentation}, funded by Kone \\Foundation. Freiburg Research Group in Saami Studies\\
\years{2013}PI in project \textbf{Down River Vashka}, Komi (Udora dialect), funded by Kone Foundation and MinorEuRus\\
\years{2012-2013}Research assistant, University of Helsinki\\

%----------------------------------------------------------------------------------------
%	EDUCATION SECTION
%----------------------------------------------------------------------------------------

\section*{Education}

\years{2013}\textsc{MA} in Finno-Ugric linguistics, University of Helsinki\\
\years{2013}\textsc{BA} in Finno-Ugric linguistics, University of Helsinki\\
\years{fall 2011}north2north program exchange student, Syktyvkar State University

%----------------------------------------------------------------------------------------
%	GRANTS, HONORS AND AWARDS SECTION
%----------------------------------------------------------------------------------------

\section*{Grants, honors \& awards}

\years{2014}VIETS Master Thesis Prize, Finnish Association for Russian and Eastern European Studies

%----------------------------------------------------------------------------------------
%	PUBLICATIONS SECTION
%----------------------------------------------------------------------------------------

\section*{Publications and Presentations}

\subsection*{Presentations}

\years{2017}Niko Partanen: Issues with archiving linguistic data. Perspectives from work on Permic languages. Conference: Travaux de terrains anthropologiques et linguistiques : expériences et difficultés. 20, May, INALCO, Paris, France\\

\years{2017} Niko Partanen: Challenges in OCR today. Report on experiences from INEL. Conference: Electronic Writing of RF Peoples: History, Issues, and Perspectives. 17 March, 2017, Syktyvkar\\

\years{2016} Daniel Jettka, Timm Lehmberg \& Niko Partanen: Digital workflows. INEL workshop. 03. November, 2016. INEL / Hamburger Zentrum für Sprachkorpora, Hamburg.\\

\years{2016} Niko Partanen: Russian influence on Iźva Komi sibilant articulation. SLE conference, 25. August, 2016, Naples, Italy\\

\years{2015} Hanna Thiele, Niko Partanen \& Michael Rießler: Exploring constituent order variation in selected languages of the Barents Sea area. Transalpine Typology Meeting, 8–9 October, 2015, Lyon, France\\

\years{2015} Rogier Blokland, Michael Rießler, Niko Partanen \& Joshua Wilbur: A critical evaluation of past, current and future approaches in Uralic language documentation. XII International Congress for Finno-Ugric Studies, 17–20 August, 2015, Oulu, Finland\\

\years{2015} Partanen, Niko \& Saarikivi, Janne: Linguistic variation in the Karelian communities. XII International Congress for Finno-Ugric Studies, 17–20 August, 2015, Oulu, Finland\\

\years{2014} Rogier Blokland, Ciprian Gerstenberger, Marina Fedina, Niko Partanen, Michael Rießler, and Joshua Wilbur: Language documentation meets language technology. First International Workshop on Computational Linguistics for Uralic Languages. 16th January, 2015, Tromsø, Norway\\

\years{2013} Saarikivi, Janne \& Partanen, Niko: Rapid extinction or relative revitalization? Karelian and Komi language communities in comparison. International Conference on Minority Languages XIV, 11–14 September 2013, Graz, Austria\\ 

\years{2012}Partanen, Niko (2012), Poster presentation: Language shift in Komi, Sociolinguistics Symposium 19, Berlin\\

\clearpage

\subsection*{Journal articles and final papers}

\years{2016} Partanen, Niko \& Saarikivi, Janne (2016), Fragmentation of Karelian language and its community: growing variation at the threshold of the language shift. In: \emph{Toivanen, Reetta \& Saarikivi, Janne (eds.), New and Old Language Diversities, Multilingual Matters}\\

\years{2014} Rogier Blokland, Michael Rießler, Niko Partanen, Marina Fedina \& Andrei Chemyshev: Ispoľzovanie cifrovyx korpusov i komp’juternyx programm v dialektologičeskix issledovanijax: teoria i praktika. In: Khisamitdinova, F.G (ed). \emph{Räsäj xaliqtary teldäre dialektologijahynyŋ könüδäk mäśäläläre. XIV Bötä Räsäj fänni konferencija materialdary (Öfö, 20-22 nojabr’, 2014 j.). Aktuaľnye problemy dialektologii jazykov narodov Rossii: Materialy XIV Vserossijskoj naučnoj konferencii (Ufa, 20-22 nojabrja 2014 g.).} Ufa: IIJaL UNC RAN. 252–255.

\years{2013}Partanen, Niko (2013), Kahden suomalais-ugrilaisen yhteisön kielenvaihdon vertailevaa tarkastelua: Tuuksa Karjalassa ja Koigort Komissa (Comparative examination of the language shift in two Uralic speech communities: Tuuksa in Karelia and Koigort in Komi), \emph{Unpublished Master Thesis, University of Helsinki}\\

\years{2013}Partanen, Niko (2013), Kielenvaihdon tarkastelua Vuokkiniemellä ja Viteleellä: Vuosien 2004 ja 2010 kenttätyöaineistojen analyysi (Observating the language shift in Vuokkiniemi and Vitele: analysis of the fieldwork materials from 2004 and 2010), \emph{Unpublished Bachelor Thesis, University of Helsinki}\\


%------------------------------------------------

\subsection*{Books}

\years{forthcoming}Partanen, Niko \& Saarikivi, Janne (eds.) (2017), \emph{Materiaaleja karjalan kielen nykytilasta (Materials on the current state of the Karelian language)}

%------------------------------------------------

\subsection*{Language skills}

Finnish: native\\
Komi-Zyrian: very good\\
English: very good\\
Italian: good\\
Russian: good\\
Udmurt: conversational\\
Komi-Permyak: conversational\\
Swedish: conversational\\
German: conversational\\
French: actively learning\\

\subsection*{Computer skills}

\textbf{Good knowledge:} R, GiellaTekno infrastructure, LaTeX, XML, GIS tools, OCR tools\\
\textbf{Learning / actively using:} Python, audio processing methods, speech recognition

\clearpage

\subsection*{Recent fieldwork experience with Permic languages}

\years{2016} Kola Peninsula, Murmanskiy Oblast, two weeks \emph{Funded by Kone Foundation}\\ 
\years{2015} Different regions in Udmurtia, ten days \emph{No external funding}\\
\years{2015} Izhemsky rayon, Komi Republic, five days \emph{Funded by Развития фундаментальных лингвистических исследований}\\
\years{2015} Naryan-Mar, Nenetskiy Autonomous Okrug, two weeks \emph{Funded by Kone Foundation}\\
\years{2014} Izhemsky rayon, Komi Republic, two weeks \emph{Funded by Kone Foundation}\\
\years{2013} Udorsky rayon, Komi Republic, a month \emph{Funded by Kone Foundation and MinorEuRus}\\
% \years{2013} Olonetsky rayon, Republic of Karelia, a week \emph{Funded by University of Helsinki}\\
\years{2012} Koygorodsky rayon, Komi Republic, two weeks \emph{Funded by MinorEuRus}\\
% \years{2012} Olonetsky rayon, Republic of Karelia, two weeks \emph{Funded by University of Helsinki}\\
% \years{2011} Pryazhinsky rayon, Republic of Karelia, two weeks \emph{Funded by University of Helsinki}\\
% \years{2010} Olonetsky rayon, Republic of Karelia, a week \emph{Funded by University of Helsinki}

%----------------------------------------------------------------------------------------
%	FINAL FOOTER
%----------------------------------------------------------------------------------------

%\begin{center}
%{\scriptsize Last updated: \today}
%\end{center}

%----------------------------------------------------------------------------------------

\end{document}